\documentclass[nobib]{tufte-handout}
\usepackage[utf8]{inputenc}
\usepackage{csquotes}

\title{Digital Consent Architecture.\\Estonia and Finland}

\author{Andres Kütt}
%\date{28 March 2010} % without \date command, current date is supplied

%
%\geometry{showframe} % display margins for debugging page layout
%\usepackage[T1]{fontenc} % Use 8-bit encoding that has 256 glyphs
%\usepackage[utf8]{inputenc}
\usepackage{graphicx} % allow embedded images
  \setkeys{Gin}{width=\linewidth,totalheight=\textheight,keepaspectratio}
  \graphicspath{{graphics/}} % set of paths to search for images
\usepackage{amsmath}  % extended mathematics
\usepackage{booktabs} % book-quality tables
\usepackage{units}    % non-stacked fractions and better unit spacing
\usepackage{multicol} % multiple column layout facilities
\usepackage{lipsum}   % filler text
\usepackage{fancyvrb} % extended verbatim environments
  \fvset{fontsize=\normalsize}% default font size for fancy-verbatim environments


% Standardize command font styles and environments
\newcommand{\doccmd}[1]{\texttt{\textbackslash#1}}% command name -- adds backslash automatically
\newcommand{\docopt}[1]{\ensuremath{\langle}\textrm{\textit{#1}}\ensuremath{\rangle}}% optional command argument
\newcommand{\docarg}[1]{\textrm{\textit{#1}}}% (required) command argument
\newcommand{\docenv}[1]{\textsf{#1}}% environment name
\newcommand{\docpkg}[1]{\texttt{#1}}% package name
\newcommand{\doccls}[1]{\texttt{#1}}% document class name
\newcommand{\docclsopt}[1]{\texttt{#1}}% document class option name
\newenvironment{docspec}{\begin{quote}\noindent}{\end{quote}}% command specification environment

% Customisations



\usepackage{array,etoolbox}
\preto\tabular{\setcounter{mrn}{0}}
\newcounter{mrn}
\def\rownumber{}


%\usepackage[
%type={CC},
%modifier={by-sa},
%version={3.0},
%]{doclicense}
\begin{document}

\maketitle% this prints the handout title, author, and date

\begin{abstract}
\noindent
The document summarises the key ideas of privacy-preserving data sharing between public and private sector institutions. They are set in the context of consensual sharing of patient information in Estonian e-health ecosystem. The solution draws heavily on the Finnish MyData concept but diverges in terms of the proposed technical solution.

\end{abstract}
\section{License}
\doclicenseThis

\section{Background}
Although in many cases\sidenote{In both Estonian and Finnish legal context agencies can have a blanket permission to access data if they are permitted to ask the same data from the customer} the legal framework in place provides a solid foundation to privacy-respecting data\sidenote{Hereinafter \enquote{data} and \enquote{APIs that give access to data or services} are used synonymously} access, a blanket consent is not very granular. By definition, such blanket consent is put in place where there is reason to believe it is in the best interest of the overwhelming majority of the citizens. Where such a reason does not exist\sidenote{E.g. in most cases where private sector would be accessing data held by the public sector; areas that are restrictive by default like genetics and healthcare; areas covered by specific data protection measures like finance and tax etc.}, blanket consent typically does not exist and data is not shared. This is often not in the interest of neither the citizen nor the organisations accessing or hosting the data. 

Moreover, a blanket consent can be controversial in nature as it forces a tradeoff between the interests of a small majority and a vast majority and the decision to provide it is inherently subjective. 

Finally, the blanket consent is by design slow to respond to the changing environment as it should not be granted easily.

Therefore, a mechanism needs to be put in place that allows for granular consenting of data access or denial thereof\sidenote{An \enquote{negative consent} is simply a reverse of what a consent is. Instead of all data being inaccessible unless a consent is in place, all data is accessible unless access is explicitly denied} by the data proprietor. 

Description of that mechanism constitutes the core of this document. 

\section{The idea}
The basic idea here is to give citizens \emph{control} over the data organisations hold about them\sidenote{This assumes data proprietors share a citizen identity: otherwise it would be impossible to map a person to the data being held}. This is explicitly different from the idea of giving citizens their \emph{data}. The  difference stems from the following assumptions:
\begin{itemize}
	\item The regular need for citizens to download their data reduces data fidelity (e.g. if a user downloads their electricity meter reading once a day, they won't get intra-day movements)
	\item The burden of having to both retain and protect their data is something an average citizen can handle worse than an organisation legally obliged to do so
	\item Increasing the number of copies\sidenote{If a data point is to be re-used by another organisation, three copies of it would necessarily be created for the download scenario: the original source of the data, the copy of the citizen and the copy of the organisation using it}  of a data point exponentially increases the chance of a breach
\end{itemize}

\section{The solution}
The basic idea is to build a multi-centred\sidenote{A number of service providers implementing standardised APIs} trust authority that maintains a list of active authorisations\sidenote{A blockchain would require less centralised trust than a database but would reduce the privacy guarantees as authorisations would essentially be public. The fact of authorisation itself, however, can be considered private information} for data access, each digitally signed by the citizen.

A common service flow would look like so:
\begin{enumerate}
	\item The user expresses the desire for an app to access their private data or service hosted by the data proprietor\sidenote{To differentiate from the data owner, the citizen, the organisation responsible for collecting and securing the data is called an \enquote{proprietor}} 
	\item The app sends a user to a trust service of their choice
	\item At the trust service, the user signs an authorisation for the app to access their data
	\item The user is sent back to the app with a token identifying the act of authorisation
	\item The app shows up at the data proprietor with a request for data or service and a token
	\item The data proprietor validates the token against the trust service and makes sure the authorisation matches the data requested
	\item The data proprietor shares the data with the app
\end{enumerate}

Figure \ref{fig:seq} illustrates the interaction in detail.

\begin{figure}[h]
  \includegraphics[width=\linewidth]{Sekvents}
  \caption{A technical sequence diagram of the authorisation process}
  \label{fig:seq}
\end{figure}

\section{Key considerations}
The solution raises the following considerations
\begin{itemize}
	\item Albeit only sharing data for immediate consumption, the data consumer can store data for unauthorised processing\sidenote{Contrary to the download model, two copies are created (the proprietor and the app) instead of three. Also, the copy is created not on a regular basis but on demand}
	\item Since the citizen authorises the transaction, they must take the full responsibility of assuring the organisation authorised has the necessary means of protecting the data shared with them
	\item A central point of trust is created that can be breached. However
	\begin{itemize}
		\item The data is not immediately lost as it still resides with the proprietor and data access can be protected by additional technical and legal means (i.e. breaching both the trust server \emph{and} an organisation the data is shared with becomes necessary)
		\item The authorisations can be secured via a personal digital signature of the citizen
	\end{itemize}
\end{itemize}

\section{Implementation}
\begin{figure*}
  \includegraphics[width=\linewidth]{Interaction}
  \caption{A detailed technical interaction diagram of the authorisation process}
  \label{fig:int}
\end{figure*}
An implementation of the solution does not exist as of \today. However, there is an agreement between a number of parties (including Estonian e-health data proprietor, technology providers, a startup and the IT arm of Estonian Ministry of Social Affairs, etc.) on a technical system design that would implement such a flow.

The current proposed solution assumes an underlying transport layer with the following capabilities
\begin{itemize}
	\item End to end encryption of all traffic (i.e. no third party can access the data shared)
	\item Authentication and explicit access rights to the data sharing APIs (i.e. data is only shared with known organisations whose identity matches the one on the authorisation artefact)
	\item Non-repudiation and audit logging (i.e. neither data proprietor, the trust service nor the service provider can claim a particular interaction did not take place)
\end{itemize}

A number of ways to build such a transport layer can be devised, the proposed solution uses X-Road\sidenote{See \url{https://www.niis.org/data-exchange-layer-x-road/}} simply because of its ready availability and low friction in the given context.

Figure \ref{fig:int} contains the proposed implementation along with a rudimentary data model. The following business processes are missing from the model:

\begin{itemize}
	\item Establishment of the service list. This should be agreed upon between the trust service, service provider and the data provider\sidenote{And published in a machine-readabale standardised format to ease interoperability} as each plays a role: 
	\begin{itemize}
		\item The trust service, as a neutral third party, presents the user with the full legal explanation of what the service entails based on input from the data provider
		\item The data provider must implement a technical API that matches the description presented by the trust service
		\item The service provider must consume the API
	\end{itemize}
	\item Audit, billing etc. that are based on the consent ID passed around for this explicit purpose\sidenote{It can be dropped from all interactions without loss of immediate functionality} and the underlying transport layer
	\item Failure of trust by 
	\begin{itemize}
		\item The trust service\sidenote{This implies existence of a supervision authority}, at which point all data providers \emph{must} stop accepting consents from that provider
		\item The service provider, at which point the service provider \emph{must} loose access to the transport layer\sidenote{This implies a mechanism to monitor service provider behaviour and a readiness by the transport layer operator to accept external requests for measures like identity revocation}  
	\end{itemize}
\end{itemize}

\end{document}
